\documentclass{beamer}
\usetheme{Warsaw}
\usepackage[spanish]{babel}
\usepackage{listings}
\lstset{basicstyle=\ttfamily}

\begin{document}
\title{Investigación en el municipio Plaza de la Revolución}
\author{Jennifer Rodríguez Moronta}
\institute{Universidad  de La Habana}
\date{\today}

\maketitle
\section{Introducción}
\begin{frame}{Introducción}
    \only<1>{Al realizar esta investigación, pude obtener valiosos conocimientos sobre las tendencias 
de precios, identificar cambios de oferta y demanda. Al hacer 
visualizaciones de la información recolectada, pude representar los datos de una manera más
intuitiva y comprensible. Las visualizaciones te permiten identificar patrones, comparar precios y
observar cambios estacionales o económicos que puedan influir en los precios.}

\only<2>{Mi investigación se enfocó 
en el municipio Plaza de la Revolución, esta tarea no fue fácil, pero me 
ayudó a crecer como estudiante. Después de caminar varios días buscando agros y cafeterías, de
 soportar los malos tratos de los dueños de los lugares ya que no entendían nuestra investigación, 
 pude llegar a la recopilación de los siguientes datos.}
\end{frame}
\section{Cerveza}
\begin{frame}{Cerveza}
    \only<1>{Lo primero es almacenar en listas la información recopilada en el JSON sobre la cerveza. Luego 
    utilizando toda esta información podemos representar los datos en gráficas; para el proceso de graficar
    utilicé la librería \texttt{matplotlib}.}
    \pause
    Las gráficas que realicé fueron sobre:
    \only<2>{
    \begin{itemize}
        \item \textbf{Porciento de alcohol en cada marca:} analizar el \% de alcohol resulta el principal control de calidad y estabilidad de las bebidas, en la recopilación de la 
    información pude observar como las personas se fijaban demasiado en el \% de alcohol de las cervezas.
       \item \textbf{Contenido en ml por marca:} si estás tratando de moderar la cantidad de alcohol
         que consumes, conocer la cantidad precisa te ayuda a establecer límites y controlar la 
         ingesta. Al conocer el contenido en ml de una cerveza, puedes estimar el número de calorías 
         que estás consumiendo.
    \end{itemize}
    }
    \only<3>{
     \begin{itemize}
         \item \textbf{Países de procedencia de la cerveza:} al analizar las estadísticas, puedes identificar las marcas de cerveza más populares en cada país. 
    Esto te brinda información sobre las preferencias de los consumidores y las tendencias actuales en 
    la industria cervecera.
        \item \textbf{Promedio de Precios de cada cerveza:}analizar el promedio de precios también te ayuda a evaluar la relación entre la calidad y el precio 
     de cada marca, puedes comparar los precios con la reputación y la satisfacción general de los 
     consumidores para determinar si una marca en particular ofrece un buen valor por el dinero que pagas.
    \end{itemize}
    }
\end{frame}
\section{Cebolla}
\begin{frame}{Cebolla}
    \only<1>{Lo primero es almacenar en listas la información recopilada en el JSON sobre la cebolla. Luego 
    utilizando toda esta información podemos representar los datos en gráficas; para el proceso de graficar
    utilicé la librería \texttt{matplotlib}.}
    \pause
    Las gráficas que realicé fueron sobre:
    \begin{itemize}
        \item \textbf{Promedio de Precios por Tipo de Cebolla:} graficar el promedio de precios de la cebolla puede proporcionar 
        información valiosa sobre su costo en el mercado, esto te permite planificar tus compras de manera más efectiva. Visualizar el promedio 
    de precios de la cebolla te permite comparar los precios de diferentes establecimientos o proveedores.
     Puedes identificar dónde puedes encontrar la cebolla a un precio más competitivo y tomar decisiones 
     en función de tus necesidades y preferencias.
    \end{itemize}
\end{frame}
\section{Malta}
\begin{frame}{Malta}
    \only<1>{Lo primero es almacenar en listas la información recopilada en el JSON sobre la malta. Luego 
    utilizando toda esta información podemos representar los datos en gráficas; para el proceso de graficar
    utilicé la librería \texttt{matplotlib}.}
    \pause
    \begin{itemize}
        \item \textbf{Contenido de ml por marca:} al graficar el contenido en ml de cada malta, puedes comparar diferentes marcas y lotes para evaluar 
    la consistencia en la cantidad de malta proporcionada.
    
    \item \textbf{Países de procedencia de la Malta:} al graficar las estadísticas de la cantidad de marcas de malta por países puedes tener una idea de la
     diversidad de opciones que existen, esto te permite explorar diferentes sabores, estilos y perfiles 
     de malta.
    
     \item \textbf{Promedio de Precios de cada Malta:} hacer este análisis te permite realizar comparaciones y 
     evaluar las opciones disponibles en el mercado, puedes identificar marcas de malta que ofrecen 
     una buena relación calidad-precio y descubrir si una marca en particular se considera más cara o
     más asequible en comparación con otras.
    
    \end{itemize}
\end{frame}
\end{document}